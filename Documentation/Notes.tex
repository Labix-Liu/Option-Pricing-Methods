\documentclass[a4paper]{article}

\input{C:/Users/liula/Desktop/Latex/Headers V1.2.tex}

\pagestyle{fancy}
\fancyhf{}
\rhead{Labix}
\lhead{Probability Theory}
\rfoot{\thepage}


\title{Probability Theory}

\author{Labix}

\date{\today}
\begin{document}
\maketitle
\begin{abstract}
\end{abstract}
\tableofcontents

\pagebreak
\section{Mathematical Finance}
A very dumbed down version of some jargons. 

Assets: Items\\
Stocks: Makes you own a small percentage of a company. \\
Shares: The unit for stocks. (Owning share = Owning a certain number of stocks)\\
Bonds: Buying debt from a loan issuer. \\
Options: Contract that gives the ability to buy / sell assets at a certain time in the future. \\
Futures: Contract to buy / sell assets at a certain time in the future. \\
Value: Total gain / loss from the whole ordeal. \\

Derivatives: Just think of them as options / future for now. \\
Strike price: the cost of exercising the derivative at the expiration date. \\

Idea: Derivatives themselves can also be traded. \\

Call: Looking to buy. \\
Put: Looking to sell. \\

European options: can only exercise at the specified date. \\
American options: can exercise any time before the specified date. \\

Derivative Traders: \\
Hedger: Actually exercises the derivative. (To minimize loss)\\
Speculator: Actually bets on the ups and downs of the derivatives. \\
Arbitrageurs: Finding certain entry point to guarantee some winnings by\\ calling / putting a combination of derivatives. (Too powerful and rare so we usually assume no arbitrage opportunities occur)\\

The price of an option is a function of 
\begin{itemize}
\item $S_0$: Current stock price. 
\item $K$: Strike price. 
\item $T$: Time to expiration. 
\item $r$: Risk free interest rate. 
\item Dividends that are expected to be paid. 
\end{itemize}
More notation: 
\begin{itemize}
\item $S_T$: Stock price at expiration day. 
\item $C/P$: American call / put. 
\item $c/p$: European call / put. 
\end{itemize}

We assume no arbitrage opportunities (if there is then it disappears quickly). 

We have 
\begin{itemize}
\item $C\leq S_0$
\item $S_0-Ke^{rT}\leq c\leq S_0$. 
\item $P\leq K$
\item $Ke^{-rT}-S_0\leq p\leq Ke^{-rT}$
\end{itemize}

More:
\begin{itemize}
\item $c+Ke^{-rT}=p+S_0$ (idea: European call and put options worth the same price at time $T$)
\item $S_0-K\leq C-P\leq S_0-Ke^{-rT}$
\end{itemize}



\pagebreak
\section{Mathematical Models for Option Pricing}
\subsection{Background Terminology}
Risk-neutral: A situation where investors do not expect an increase in return when the risk increases. This has two consequences. 
\begin{itemize}
\item The expected return of a stock is the risk free rate. 
\item The discount rate for the expected pay off of an option is the risk free. 
\end{itemize}

Risk-less: A portfolio is risk-less when regardless of the outcome, the net total is constant. 

\subsection{One-Period Binomial Models}
\begin{defn}{The Binomial Model}{} Suppose that $t$ is the current time. An asset has a current stock price of $S_0$ with an unknown option price $\mO$. After a time step of $dt$, the stock price either increase by a multiplicative factor of $u>1$ or the stock price decreases by a multiplicative factor of $d<1$. Denote the corresponding option value (the net gain from the option) by $\mO_u$ and $\mO_d$ respectively. \\~\\
\adjustbox{scale=1.0,center}{\begin{tikzcd}
	t && {t+dt} \\
	&& \begin{array}{c} \substack{uS_0\\\mO_u} \end{array} \\
	\begin{array}{c} \substack{S_0\\\mO} \end{array} \\
	&& \begin{array}{c} \substack{dS_0\\\mO_d} \end{array}
	\arrow["dt", from=1-1, to=1-3]
	\arrow["p", from=3-1, to=2-3]
	\arrow["{1-p}"', from=3-1, to=4-3]
\end{tikzcd}} \\~\\
The number $0<p<1$ is an unknown quantity denoting the probability that the stock price increases. 
\end{defn}

\begin{lmm}{}{} Assume there is no arbitrage opportunity and the world is risk-neutral. Assume the binomial model. Suppose a portfolio of buying $\Delta\in\N$ stocks and shorting a call option. Then the portfolio is risk-less when $$\Delta=\frac{\mO_u-\mO_d}{uS_0-dS_0}$$
\end{lmm}

\begin{prp}{}{} Assume there is no arbitrage opportunity and the world is risk-neutral. Assume the binomial model. Denote $V(t)=S_0\Delta-\mO$ the value of the portfolio of buying $\Delta$ stocks and shorting a call option. Then the following are true. 
\begin{itemize}
\item The probability $p$ that the stock price increases is given by $$p=\frac{e^{rdt}-d}{u-d}$$
\item The option price is given by $$\mO=e^{-rdt}\left(p\mO_u+(1-p)\mO_d\right)$$
\end{itemize} \tcbline
\begin{proof}
Accounting for the discount rate, we have $$V(t+dt)=(uS_0\Delta-\mO_u)e^{-rdt}=(dS_0\Delta-\mO_d)e^{-rdt}$$ by definition of $\Delta$. Equating $V(t)$ and $V(t+dt)$ gives the desired result. 
\end{proof}
\end{prp}

\begin{lmm}{}{} Assume there is no arbitrage opportunity and the world is risk-neutral. Assume the binomial model. Suppose that the implied volatility is $\sigma$. Then we have $$u=e^{\sigma\sqrt{dt}}\;\;\;\;\text{ and }\;\;\;\;d=e^{-\sigma\sqrt{dt}}$$
\end{lmm}

\subsection{Multi-Period Binomial Models}
\begin{defn}{The Multi-Period Binomial Model}{} Suppose that $t$ is the current time. An asset has a current stock price of $S_0$ with an unknown option price $\mO$. After a time step of $dt$, the stock price either increase by a multiplicative factor of $u>1$ or the stock price decreases by a multiplicative factor of $d<1$. Denote the corresponding option value (the net gain from the option) by $\mO_u$ and $\mO_d$ respectively. \\~\\
\adjustbox{scale=1.0,center}{\begin{tikzcd}
	&&&& \begin{array}{c} \substack{u^2S_0\\\mO_{u^2}} \end{array} & \cdots \\
	&& \begin{array}{c} \substack{uS_0\\\mO_u} \end{array} \\
	\begin{array}{c} \substack{S_0\\\mO} \end{array} &&&& \begin{array}{c} \substack{udS_0\\\mO_{ud}} \end{array} & \cdots \\
	&& \begin{array}{c} \substack{dS_0\\\mO_d} \end{array} \\
	&&&& \begin{array}{c} \substack{d^2S_0\\\mO_{d^2}} \end{array} & \cdots
	\arrow["p", from=2-3, to=1-5]
	\arrow["{1-p}", from=2-3, to=3-5]
	\arrow["p", from=3-1, to=2-3]
	\arrow["{1-p}"', from=3-1, to=4-3]
	\arrow["p"', from=4-3, to=3-5]
	\arrow["{1-p}"', from=4-3, to=5-5]
\end{tikzcd}} \\~\\
The number $0<p<1$ is an unknown quantity denoting the probability that the stock price increases. The process iterates for $n$-times. 
\end{defn}

Notice at each time-step, the process is identical except with a different current stock and option price. Therefore we can iterate the one-period binomial model $n$-times to recover the option value. 

\begin{prp}{}{} Assume there is no arbitrage opportunity and the world is risk-neutral. Assume the mutli-period binomial model with period $n$. Then the European call option price is given by $$\mO=e^{-nrdt}\left(\sum_{i=0}^n\binom{n}{i}p^{n-i}(1-p)^i\mO_{u^{n-i}d^i}\right)$$
\end{prp}

\subsection{Black-Scholes-Merton Model}
\begin{defn}{Black-Scholes-Merton Model}{} Let $T$ be a set time (in years). Denote $S:\Omega\times[0,T]\to\R$ the stochastic processes that is the value of a certain stock. Let $\mu$ be the annual expected return. Let $\sigma$ be the annual volatility of the stock price. The Black-Scholes-Merton Model makes the following assumptions
\begin{itemize}
\item The percentage in stock is normally distributed: $$\frac{S_{t+dt}-S_t}{S_t}\sim N(\mu dt, \sigma^2dt)$$ 
\item $\mu$ and $\sigma$ are constants over time. 
\item The risk free interest rate is constant over time. 
\item There are no dividends during the lifetime of the stock. 
\item There are no arbitrage opportunities. 
\end{itemize}
\end{defn}

\begin{lmm}{}{} Let $T$ be a set time (in years). Denote $S:\Omega\times[0,T]\to\R$ the stochastic processes that is the value of a certain stock. Let $\mu$ be the annual expected return. Let $\sigma$ be the volatility of the stock price. Assume the Black-Scholes-Merton model. Then we have $$\ln(S_t)\sim N\left(\ln(S_0)+\left(\mu-\frac{\sigma^2}{2}\right)T,\sigma^2T\right)$$
\end{lmm}

Rate of return: The percentage return averaged over a certain amount of time. 

\begin{lmm}{}{} Let $T$ be a set time (in years). Denote $S:\Omega\times[0,T]\to\R$ the stochastic processes that is the value of a certain stock. Assume the Black-Scholes-Merton model. Let $r$ be the continuously compounded rate of return of $S$. Then we have $$r\sim N\left(\mu-\frac{\sigma^2}{2},\frac{\sigma^2}{T}\right)$$
\end{lmm}

Volatility: a measure of uncertainty of return. So it is precisely the variance of the rate of return. Hence the annual volatility is $\sigma$. Checks out. 

\begin{prp}{}{} Let $T$ be a set time (in years). Denote $S:\Omega\times[0,T]\to\R$ the stochastic processes that is the value of a certain stock. Assume the Black-Scholes-Merton model. Then the European call option price according to the model is $$c=S_0 F_N(d_1)-Ke^{-rT}F_N(d_2)$$ where $K$ is the strike price, $r$ is the risk-free interest rate and $F_N$ is the cumulative distribution function of $N(0,1)$ and $$d_1=\frac{\ln(S_0/K)+(r+\sigma^2/2)T}{\sigma\sqrt{T}}\;\;\;\;\text{ and }\;\;\;\;d_2=\frac{\ln(S_0/K)+(r-\sigma^2/2)T}{\sigma\sqrt{T}}=d_1-\sigma\sqrt{T}$$
\end{prp}











\end{document}